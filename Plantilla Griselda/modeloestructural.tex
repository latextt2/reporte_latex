\section{Modelo Estructural}\index{Modelado!Modelo Estructural}

El dise�o se hizo con la ayuda de algunas referencias, como una tesis de la Universidad Libre de Berl�n en Alemania, en donde se reporta un proyecto muy parecido al presentado en esta tesis, otras referencias son kits de robots que se venden comercialmente y que presentan algunas de las  caracter�sticas que se plantean en el proyecto.\\

La elecci�n en cuanto al material fue el aluminio, ya que presenta caracter�sticas que  benefician la etapa de construcci�n, por ejemplo: es maleable, f�cil de maquinar, ya sea en un maquinado con arranque de viruta o sin arranque de viruta; su densidad es baja, lo que resulta en un peso bajo, es f�cil de conseguir y relativamente barato.\\
