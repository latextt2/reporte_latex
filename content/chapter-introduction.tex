% !TEX root = ../thesis-example.tex
%
\chapter{Introducción}			%El [*] para no enumerar el capitulo en la tabla de contenidos
\label{sec:intro}
\addcontentsline{toc}{chapter}{Introducción}	%Para que el capitulo aparezca en la tabla de cont.

\cleanchapterquote{You can’t do better design with a computer, but you can speed up your work enormously.}{Wim Crouwel}{(Graphic designer and typographer)}

El desarrollo de nuevas investigaciones y nuevas tecnologías han permitido que la industria busque nuevas y mejores formas de llevar a cabo sus procesos productivos, lo que requiere, entre otras cosas, de transformar y actualizar la maquinaria y equipo que utilizan con el propósito de volverlos más eficientes y aumentar su productividad.
En este aspecto se han realizado numerosos avances y descubrimientos, como pueden ser el uso de fuentes de energía más limpias y eficientes, la implementación de nuevos materiales más ligeros y resistentes, y por supuesto el reemplazo de algunos componentes y dispositivos por otros más avanzados.
Gran parte de la maquinaria industrial posee elementos de tipo rotativo, los cuales requieren el uso de dispositivos denominados cojinetes, también conocidos como rodamientos o baleros. Estos suelen situarse entre dos componentes de la maquina con un eje de rotación común, facilitando el movimiento de giro de un componente con respecto al otro y reduciendo la fricción de los diferentes elementos móviles; son a su vez, un punto de apoyo para dichos ejes o árboles. 
Además de ser capaces de disminuir la fricción, de manera general los rodamientos deben cumplir con algunas características particulares, como son: disminuir las vibraciones, operar de manera silenciosa y poseer una prolongada vida útil. Sin embargo, los distintos sectores industriales suelen establecer sus propios requisitos específicos con base en determinados entornos operativos o aplicaciones particulares, entre los que podemos mencionar: resistir grandes cambios de temperatura, humedad, suciedad, operar en medios agresivos como soluciones acidas o alcalinas, soportar altas velocidades e incluso ser aptos para el contacto con alimentos [1]. 
Es por ello que dentro de los rodamientos existe una gran variedad de configuraciones, materiales y modos de operación que puedan satisfacer esta demanda de necesidades.


\section*{Clasificación de Cojinetes y Rodamientos}
\label{sec:intro:address}
\addcontentsline{toc}{section}{Clasificación de Cojinetes y Rodamientos}

Los cojinetes se clasifican principalmente en dos grandes grupos: cojinetes de deslizamiento (o de fricción) y antifricción (rodamientos). Los cojinetes de deslizamiento consisten básicamente en un par de casquillos concéntricos que giran en contacto directo uno con el otro realizándose un deslizamiento por fricción, procurando que esta sea la menor posible; la disminución de la fricción depende de los materiales con los que están fabricados y del uso de alguna película lubricante [2]. 
Por su parte, los cojinetes antifricción (rodamientos) comprenden una clasificación más extensa debido a su complejidad estructural, en comparación a los cojinetes de deslizamiento. Entre los principales podemos encontrar: por la dirección de la carga que soportan (axial, radial, angular o mixta, y lineal), según su rigidez (rígidos y pivotantes) y por el tipo de cuerpo rodante (de bolas, rodillos, agujas, cónicos, etc.) [3]. 
El rozamiento por rodadura generado por los rodamientos es mucho menor que el de los cojinetes de deslizamiento, y es por ello que presentan una serie de ventajas más frente a la utilización de casquillos que incluye mayor velocidad admisible, menor temperatura de funcionamiento, mayor capacidad de carga, menor desgaste, facilidad de recambio y disminución en los costos de mantenimiento. 
A pesar de todo esto, existen algunas aplicaciones para las cuales se deben buscar alternativas a estos dispositivos mecánicos, particularmente las que están relacionadas con máquinas rotativas de alta velocidad donde se incrementan de manera considerable las perdidas mecánicas, las cuales están asociadas al rozamiento y a las vibraciones mecánicas. Ante este problema surge un tipo de rodamiento que opera por medio de la suspensión magnética, conocidos como rodamientos magnéticos.  

\section*{Rodamientos Magnéticos}
\label{sec:intro:motivation}
\addcontentsline{toc}{section}{Rodamientos Magnéticos}

Como ya se mencionó anteriormente, los rodamientos magnéticos son una excelente alternativa para su uso en máquinas rotativas de alta velocidad, ya que presentan múltiples ventajas frente a los rodamientos tradicionales, como una mayor velocidad de giro, no requieren de un sistema de lubricación ya que operan sin contacto, por lo que además no producen partículas de desgaste y debido a su baja rigidez son capaces de compensar las vibraciones mecánicas, haciéndolos silenciosos. Todo esto se traduce en una disminución de los costos de mantenimiento y prolongación de la vida útil del sistema. 
El potencial que poseen para generar muy pocas pérdidas y alcanzar mayores velocidades los hace ideales para aplicaciones industriales como centrifugadoras de alta velocidad, volantes inerciales, compresores y turbinas de alto régimen de revoluciones, por mencionar algunas. 
Los rodamientos magnéticos cuentan, a su vez, con una clasificación que viene en función de dos principales factores: de acuerdo a la naturaleza de su fuerza, que puede ser de Reluctancia o de Fuerza de Lorentz, o más comúnmente de acuerdo a la fuente generadora del campo magnético (Pasivos, Activos e Híbridos) [4]. 
Los pasivos carecen de alimentación externa y emplean imanes permanentes para generar los campos magnéticos de levitación, mientras que los activos utilizan electroimanes y requieren de una fuente de alimentación externa y de un sistema de control. Por su parte, los híbridos emplean imanes permanentes para crear un campo base de levitación mientras que el resto del campo es generado por electroimanes.
A pesar de que la mayoría de los usos que se le dan a los rodamientos magnéticos pertenecen al sector aeroespacial, naval, generador de energía y otras aplicaciones de alta demanda, también es posible aprovecharlos en otro tipo de maquinaria más comercial pudiendo explotar al máximo sus beneficios. 


\section*{Antecedentes}
\subsection*{Investigación y Desarrollo de Rodamientos Magnéticos}
\label{sec:intro:results}
\addcontentsline{toc}{section}{Antecedentes}
\addcontentsline{toc}{subsection}{Investigación y Desarrollo de Rodamientos Magnéticos}

Investigación y Desarrollo de Rodamientos Magnéticos 
A pesar de las múltiples ventajas que presentan, los rodamientos magnéticos no están exentos de algunos problemas que pueden dificultar su desarrollo e implementación en los sistemas industriales, entre los que podemos destacar el alto consumo energético que generan.
Ante este problema se hicieron varias investigaciones cuyas soluciones pueden clasificarse en dos tipos: el diseño de algoritmos para minimizar las corrientes en los electroimanes y el uso de imanes permanentes para proveer un campo magnético base para la levitación.
La investigación realizada en [5] se trata de una validación experimental en donde 1) se diseña y construye un prototipo de pruebas que emula el comportamiento de un rodamiento magnético activo de un grado de libertad, y 2) se realizaron pruebas para evaluar el desempeño del controlador inteligente en comparación con un controlador estándar de polarización fija.
Este equipo consiste en un eje rígido capaz de balancearse libremente sobre un punto de apoyo colocado debajo de su centro de masa. La posición angular de este eje es controlada mediante electroimanes fijados a los extremos de este. A pesar su simplicidad este equipo incorpora todas las no linealidades típicas de un rodamiento magnético activo, y la dinámica del sistema es ideal para evaluar el desempeño de los controladores. 
En [6] una estrategia de control de corriente de polarización variable es presentada con el fin de minimizar la energía consumida sin alterar el desempeño dinámico del sistema. El modelado parte de un RMA de 2 polos sobre el cual se propone un controlador lineal y un observador de estados para la estimación de estados.
Una vez obtenido el modelo del sistema se calcula de manera analítica la corriente óptima para la estabilización del sistema y obtener el controlador. 

Como resultado este controlador propuesto fue capaz de reducir el consumo en un 73\%, se estabilizó la posición de rotor en el centro del rodamiento con un error de estado estable nulo y se presentaron vibraciones 60\% menores comparadas con un controlador de polarización fija.

\subsection*{Rodamientos Magnéticos en la Industria}
\label{sec:intro:results}
\addcontentsline{toc}{subsection}{Rodamientos Magnéticos en la Industria}

En la actualidad los sistemas de rodamientos magnéticos tienen múltiples aplicaciones dentro del campo industrial, y aunque la mayoría cuenta con adaptaciones acorde a la maquinaria en la que se utilizará, poseen una arquitectura y un modo de funcionamiento similar.
La empresa FAG®, filial de Schaeffler Group, ha desarrollado un sistema de rodamiento magnético activo que denominan como modular. Su sistema de control y de electrónica de potencia es capaz de ajustarse a ciertos parámetros definidos por el usuario acorde a los requerimientos de operación, y su diseño permite una fácil integración en la arquitectura de la máquina.
Otra de las ventajas que posee es que prácticamente no existen restricciones de velocidad y puede soportar pesos de eje de más de 9 toneladas. Cuenta con un sistema de apagado de seguridad que previene daños a la máquina y gracias a los cojinetes de respaldo que integra también la protege en caso de una interrupción en el suministro de energía que conduzca al fallo del cojinete magnético. A su vez sirven de apoyo para el rotor cuando los cojinetes magnéticos están apagados [7].
Sin embargo, ya que se trata de una empresa especializada este producto resulta altamente costoso, además de que únicamente están orientados a aplicaciones de alta demanda y cargas muy grandes.
Por su parte la empresa YORK® Navy Systems de Johnson Controls, quien desarrolló uno de los primeros sistemas de rodamientos magnéticos para su uso en enfriadores centrífugos en embarcaciones de grado militar, ha sido capaz de aplicar esta tecnología en enfriadores comerciales, logrando alargar el tiempo de vida de la máquina disminuyendo el número de partes móviles y removiendo los sistemas de lubricación tradicionales. Esto se traduce en una reducción significativa de los costos de mantenimiento.
Utiliza un sistema de velocidad variable el cual ralentiza el motor cuando el enfriador está funcionando en condiciones fuera de diseño, además de mejorar la fiabilidad durante el arranque al asegurar que la corriente de entrada nunca supere el 100% de los amperes de carga completa. Y finalmente gracias a las vibraciones casi indetectables, el ruido se reduce casi por completo [8].
Aunque han logrado ampliar esta tecnología para aplicaciones comerciales e industriales fuero del ámbito militar, este sistema está desarrollado casi exclusivamente para plantas de enfriamiento y aire acondicionado, por lo que no es capaz de usarse y adaptarse a otro tipo de sistemas.
Por otro lado, la empresa Synchrony® ha realizado grandes avances en cuanto a tecnología de rodamientos magnéticos se refiere. Han logrado disminuir el tamaño de la unidad de control eliminando o miniaturizado algunos componentes como los convertidores A/D y D/A, los amplificadores de potencia y la interfaz de comunicaciones, y el uso de contadores de alta velocidad para la señal de posición. Una de las innovaciones más importantes ha sido el desarrollo de sensores de posición que pueden ser integrados directamente en los electroimanes.
Han reducido dramáticamente el tamaño de los rodamientos magnéticos axiales y radiales mediante el uso de nuevas e innovadoras técnicas de diseño y manufactura. Todas estas modificaciones han reducido en general la complejidad del sistema [9].
Sin embargo, la disponibilidad de productos es baja, y solamente ofrecen servicio a un número muy limitado de países.

\section*{Planteamiento del problema}
\label{sec:intro:results}
\addcontentsline{toc}{section}{Planteamiento del problema}

En la actualidad, los avances tecnológicos tienden hacia el desarrollo de sistemas más pequeños, veloces, eficientes, y con un menor consumo de energía y mínimo mantenimiento. Estas cualidades adquieren gran importancia dentro del ámbito industrial, donde sus beneficios impactan directamente en el desarrollo de mejores máquinas y dispositivos, como motores, compresores, enfriadores, sistemas de centrifugación y sistemas de transmisión, por mencionar algunos.
Gran parte de estos y otros dispositivos rotativos no son capaces de alcanzar altas velocidades debido al rozamiento y a las vibraciones que se producen en los ejes y árboles de transmisión; entre más revoluciones alcance, mayor será el calor generado por la fricción, y las pérdidas mecánicas incrementan de manera considerable. 
Pese a que los rodamientos suprimen algunas de las desventajas propias de los cojinetes de deslizamiento, aún presentan algunos inconvenientes para su uso en maquinaria de alta velocidad; estos problemas abarcan principalmente su sensibilidad a la contaminación debido a la generación de partículas de desgaste, una baja resistencia a las cargas de impacto , generación de cargas fluctuantes  (vibraciones), mayor nivel de ruido acústico, y aunque el coeficiente de rozamiento es bajo, los materiales continúan funcionando en contacto directo, lo que provoca un incremento en las temperaturas de operación. 
Para hacer frente a estas desventajas es necesario considerar el uso de un nuevo tipo de rodamientos que reemplace a los convencionales, por lo que un sistema de rodamientos magnéticos resulta en una buena alternativa que atrae importantes beneficios. 
Dado que los rodamientos magnéticos trabajan sin contacto directo con los elementos que soportan, pueden disminuir de manera considerable las pérdidas mecánicas debido a la supresión del rozamiento de los cojinetes de apoyo y otras partes móviles. Esto permite reducir la potencia necesaria de los motores e incrementar la eficiencia, y son capaces de compensar las vibraciones y cargas fluctuantes. 
La reducción de las pérdidas implica también que las temperaturas de operación sean menores, y en algunos casos esto permite que no sea necesario el uso de equipos de refrigeración adicionales. 
Los rodamientos magnéticos no requieren del uso de lubricantes, lo que los hace ideales para máquinas que trabajan en ambientes con temperaturas extremas, en condiciones de vacío, que se encuentren en contacto con sustancias corrosivas o cualquier máquina que no tolere la contaminación por las partículas de desgaste o de los mismos lubricantes, además elimina la necesidad de implementar sistemas de lubricación extras.
Todos estos beneficios implican una reducción significativa de los costos inherentes al mantenimiento, relacionados principalmente al reemplazo de piezas debido al desgaste, además de la eliminación de sistemas adicionales de refrigeración y lubricación y a una necesidad de mantenimiento periódico menor [10]. 
A pesar de las múltiples ventajas que ofrece este tipo de sistemas, no se ha logrado ampliar su uso dentro de la industria debido principalmente a que presentan otro tipo de desventajas que no se encuentran en los cojinetes y rodamientos, como su accesibilidad limitada, un precio más elevado, la necesidad de contar con un sistema de control y la posibilidad de fallas en el sistema eléctrico, además de que en el caso de los rodamientos magnéticos activos el consumo eléctrico suele ser elevado. Es por eso que en este sentido se optó por el desarrollo de rodamientos magnéticos de funcionamiento híbrido, ya que al utilizar un campo base otorgado por los imanes permanentes, el consumo eléctrico de los electroimanes es menor. 
Este proyecto tendrá como finalidad desarrollar un sistema que pueda brindar una alternativa a los rodamientos magnéticos tradicionales en proyectos relacionados con el desarrollo de maquinaria y equipo industrial.


\section*{Descripción del Trabajo}
\label{sec:intro:structure}
\addcontentsline{toc}{section}{Marco de Referencia}
