

%\pdfbookmark[0]{Acknowledgement}{Acknowledgement}
\chapter*{Nomenclatura}
\label{sec:Acknowledgement}
\vspace*{-10mm}
%\renewcommand{thechapter}{\Roman{Nomenclatura}}
\addcontentsline{toc}{chapter}{Nomenclatura}

\chapter*{Unidades/Simbología}
\label{sec:Acknowledgement}
\vspace*{-10mm}
\addcontentsline{toc}{chapter}{Unidades/Simbología}

\chapter*{Objetivos}
\label{sec:Acknowledgement}
\vspace*{-10mm}
\addcontentsline{toc}{chapter}{Objetivos}

\section*{Objetivo General}
Diseñar y simular un sistema de rodamientos magnéticos capaz de operar en condiciones de carga estática.

\section*{Objetivos Específicos}

{\setlength{\parindent}{0pt}Trabajo Terminal I:}
\begin{enumerate}
\addtolength{\itemsep}{0pt}
\item Obtener el diseño conceptual del rodamiento magnético híbrido.
\item Obtener el diseño conceptual del rodamiento magnético híbrido.
\item Seleccionar el material ferromagnético para el núcleo de los electroimánes. 
\item Calcular la sección transversal de los electroimánes.
\item Obtener la geometría y dimensiones de los rodamientos activos radial y axial.
\item Validar el punto de operación del electroimán mediante simulación.
\item Diseñar un electrodo de alta sensitividad para el rango de operación.
\item Diseñar un dispositivo para la retracción del imán permanente. 
\item Proponer un circuito de adquisición y control. 
\end{enumerate}

Trabajo Terminal II:
\begin{enumerate}
\addtolength{\itemsep}{0pt}
\item Obtener un modelo de electroimánes que integre los efectos del flujo marginal.
\item Realizar las simulaciones de la fuerza magnética ejercida por los electroimánes y el imán permanente.
\item Obtener el modelo dinámico del sistema.  
\item Comparar algunos esquemas de control para la estabilización del sistema.
\item Diseño del circuito de potencia de los electroimánes. 
\item Estimar los límites de operación del rodamiento con base en las simulaciones.
\end{enumerate}
